
\section{Usage Scenarios}

To illustrate how \textit{ARLayout} searches, re-groups and re-ranks massive physical targets
in AR environments,
we describe three usage scenarios where users use \textit{ARLayout} to
search and filter targets, obtain their information, and locate
ideal ones among massive targets,
i.e., the usage scenarios in a library/bookstore, a coffee shop and an eyeshadow fitting.

\subsection{Library/Bookstore Scenario}

Suppose Zelda is a student major in economy.
She wants to buy some economic books to broaden her horizons.
She prefers books from the ``University of Chicago Press'',
which is recognized as having been publishing high-quality books.
She comes to the social science area in a library/bookstore,
facing a large row of bookshelves with around 1000 books.
Instead of looking for books in the bookstore's searching system (which is often just
open for bookstore clerks),
by which she may have dizzy and tedious choosing experience.

(1) \textbf{Fuzzy searching/filtering}: she decides to quickly scan the whole bookshelves by the panoramic camera by \textit{ARLayout}.
Soon 778 books are recognized in total.
She then filters unrelated books by saying ``economic''.
\textit{ARLayout} deals with the voice input, and filters those books by fuzzy search.
% fisheye
Seeing that only 118 economic books remains, Zelda chooses to visualize those books
in the AR space and browses them as shown in~\autoref{fig:casestudy_book} (b).
She finds that only one book nearby is from ``University of Chicago Press'',
then she want to find more books on ``economic' and published by ``University of Chicago Press''.

% re-groupp
(2) \textbf{Re-grouping}: she re-group those 118 books by publisher and searches by saying ``Chicago''
or inputting by the virtual keyboard of her mobile phone.
This time, seven books from the ``University of Chicago Press'' are highlighted and
placed on a bookshelf in front of her with fisheye effect (\autoref{fig:casestudy_book} (c)).
Books from other presses are also grouped and placed on the other layers of the
shelf (\autoref{fig:casestudy_book} (d)), so she choose a book from them.

% search
(3) \textbf{Fuzzy re-searching}: she wants to search all books with fuzzy keyword ``social'' ,
there are 21 books highlighted in red (\autoref{fig:casestudy_book} (d)).
She uses fisheye effect to view each book's details like author and reviewers' words (\autoref{fig:casestudy_book} (c)).
But she finds these social books are not highly rated or the authors are not in her favorite author list,
So she changes idea: she re-ranks those books by ratings or further filter them by author names.

% re-rank
(4) \textbf{Re-ranking}: she sort all of the books which are placed
from left to right on the same layer of the shelf by descending order (\autoref{fig:casestudy_book} (e)).
Then she selects four books that seems suitable,
those selected books are moved to a lower layer of the
virtual bookshelf where are designed to place the candidate books (\autoref{fig:casestudy_book} (f)),
just like a shopping cart.

% word cloud
(5) \textbf{Comparing by word cloud visualizations in small multiples}:
she views and compares the word cloud of each book's keywords.
Among those four books, one book has keywords ``story'' and ``understand'', other books' keywords are
``city'', ``environmentalist'' and ``empire'' (\autoref{fig:casestudy_book} (g)).
Zelda is interested in the ``story'' one and the ``empire'' one, but she's also concerned with the prices
if she is going to buy in a bookstore.


(6) \textbf{Comparing with kinds of diagrams in small multiples}:
so she compares these books' ratings and prices together
in the bar chart (\autoref{fig:casestudy_book} (h)).
She finds that the ``story'' (the first) rated as high as
the ``empire'' one (the fourth), but is a little cheaper than the ``empire'' one.
So she choose the ``story'' one and restores those books to their original layout.

(7) \textbf{Title Precise Searching}: finally, she voice input the title ``human resources management''.
The book is magnified and highlighted on the left upper side (\autoref{fig:casestudy_book} (i)) by flashing.
She walks by and locates the book in the reality space according to
its position shown in the screen (\autoref{fig:casestudy_book} (j)).




% \begin{figure}[htp]
%     \centering
%     \includegraphics[width=\linewidth]{images/books.eps}
%     \caption{
%      coffee menu.
%     }
%     \label{fig:casestudy}
% \end{figure}

% Another day, Zelda wonders in a book shop looking for interesting books to read.
% She walks to the history area and plans to choose some books about economics.
% She scans those shelves and \textit{ARLayout} recognize about 1500+ books. She then input
% "economic" by voice. After fuzzy search and filtering, 118 books remains.
% % She visualizes those books, and views their details by fisheye effect.
% However, she's not familiar with economic books' publishers and authors,
% so she re-ranks those books by ratings. A virtual shelf shows up, with books sorted by
% descending order placed from left to right.
% Then she browses books with fisheye effect from left to right,
% chooses 4 books with appealing titles.
% She compares the word cloud of those books and find that they all suit her taste.
% She then compares the ratings and prices by charts, chooses the book "" with a relatively high price–performance ratio.
% Then, similar to finding the algorithm book previously, she restores virtual books, searches the title,
% and successfully finds the ideal book in the real space.

% \begin{figure}[htp]
%     \centering
%     \includegraphics[width=\linewidth]{images/Cafe.eps}
%     \caption{
%         virtual coffee menu.
%     }
%     \label{fig:casestudy}
% \end{figure}

\subsection{Coffee Shop Scenario}

A new coffee shop opens in Zelda's campus.
she don't know much about coffee,
but is willing to try some in the new coffee shop.
She walks in the coffee shop, takes a picture of the coffee menu by \textit{ARLayout}.
Soon it recognizes 40 different drinks, and creates a menu upon the original one.
% She browses the menu using fisheye effects,

She remembers that she once ordered a cup of espresso,
which she thinks is rather bitter, so she wants to see the ingredients.
% To compare espresso with other coffees,
So she firstly voice inputs ``Espresso'' and finds that
it's highlighted in the ``Hot Coffees'' group(\autoref{fig:casestudy_coffee} (b)).
She views the details of Espresso, learns that it's surely bitter,
as no sugar is added in it (\autoref{fig:casestudy_coffee} (c)).

She then re-groups those drinks according to milk or sugar (\autoref{fig:casestudy_coffee} (d-e)).
She browses and selects some drinks with high ratings in the ``medium sweet''
and ``sweet'' group, as shown in~\autoref{fig:casestudy_coffee} (f).
Then she compares those drinks' ingredients in small multiples,
and finds that Cappuccino has a balance between sugar, milk and caffine,
which may suit to her taste, as shown in~\autoref{fig:casestudy_coffee} (g).
However, her fitness coach's advice cross her mind that she needs to limit the calorie intake to 1300 calories everyday,
while the coffee summary shows that Cappuccino has 140 calories per cup.
So she re-ranks all the drinks by calorie content.
This time, coffees are sorted from left to right by calorie,
as shown in~\autoref{fig:casestudy_coffee} (h).
She begins browsing on the right side, where coffees with relatively low calorie are located.
She find some of coffees that she haven’t drunk.
To have a quick grasp of them, she views their word cloud (\autoref{fig:casestudy_coffee} (i)).
She learns that Blonde Roast is regarded to be ``mellow'' in the word cloud,
Iced Coffee is ``rich'',
and Caffee Americano has keyword ``espresso'', which may be too bitter for her.
% She again searches the Blonde Roast, find that it locates to the right of
% the menu where coffees with relatively low calorie content are located(shown in~\autoref{fig:casestudy_coffee} (e)).
She browses Blonde Roast's summary, which confirms that it only contains five calories per cup (\autoref{fig:casestudy_coffee} (j)).
Finally, she chooses Blonde Roast and enjoys its ``soft and mellow flavor'' described in the summary.

In addition, \textit{ARLayout} can also handle larger menus like the poster hanging on the wall
outside the coffee shop, as shown in~\autoref{fig:casestudy_coffee_poster} (a).
The user can view details var fisheye, re-group, re-rank, select and compare drinks on the poster, 
which is similar to the smaller menu, as shown in~\autoref{fig:casestudy_coffee_poster} (b-d).


% Since Zelda has an enjoyable experience learning and choosing drinks in the coffee shop,
% several days later, she visits the coffee shop again.
% % , planning to explore and try something different.
% This time, she plans to explore and try something different.
% "Maybe coffees with no milk", she thinks.
% She scans the menu and starts browsing in virtual space.
% Firstly, she re-groupps the drinks according to milk content.
% In the "<=5\%" group, she selects some coffee and drinks.
% To have a quick grasp of these drinks, she views their word cloud(shown in~\autoref{fig:casestudy} (j)).
% She learns that the Iced Coffee in this coffee shop is regarded as "rich",
% Blonde Roast is "mellow",
% and Caffee Americano has keyword "espresso", which may be too bitter for her.
% Before deciding between Blonde Roast and Iced Coffee,
% she remembers her fitness coach's advice of limiting the calorie intake everyday.
% So she re-ranks all the drinks by calorie content.
% She then searches these two coffees,
% finds that Iced Coffee locates at the left side of the menu where drinks contains more calories,
% while Blonde Roast locates at the right side, indicating its relatively low calorie content(shown in~\autoref{fig:casestudy} (k)).
% The fisheye effect confirms that Blonde Roast only contains 5 calories per cup(shown in~\autoref{fig:casestudy} (l)).
% Finally, she chooses Blonde Roast and enjoys its soft and mellow flavor.


\subsection{Eyeshadow Scenario}

One day Zelda is shopping in a cosmetics shop.
She is not good at making up, especially eyeshadow,
because different eyeshadows may have unique effects,
and sometimes several eyeshadows may be applied to different places
to form a colour combination.
So she uses \textit{ARLayout} and scans those eyeshadows displayed on the desk, as shown in~\autoref{fig:casestudy_eye} (a).
Soon 15 different eyeshadows with 96 different colors (or textures) are recognized.
She then re-groups them by eyetypes, and use fisheye effect to view the details of the ``protruding eye'' group,
as shown in~\autoref{fig:casestudy_eye} (b).
% Those slightly different eyeshadows in this color group may have their unique effects.
A graph pops up on the side of the selected eyeshadow, showing the ideal position for users to apply it on.
Zelda chooses a kind of golden brown eyeshadow,
and previews its 3-D effect on a virtual model, as shown in~\autoref{fig:casestudy_eye} (c).
Zelda still finds it hard to choose several eyeshadows that matches each other,
so she re-groups them by high rated schemes, this time,
three recommended color schemes are lined up in front of her, as shown in~\autoref{fig:casestudy_eye} (d).
Zelda views the details about each eyeshadow's effects and features,
learns that eyeshadows in ``Scheme 13'' is suitable for simple day look.
So she restores those colors to their original layout and search for ``Scheme 13'' by voice.
Those eyeshadows in ``Scheme 13'' are flashing in red, as shown in~\autoref{fig:casestudy_eye} (e).
As a result, she chooses an eyeshadow palette that contains
some colors in ``Scheme 13''.
% Details about each eyeshadow's effects and features are shown on the UI.

% There's also details  about its features and style.

