\section{Design Rationale}
\label{sec:design}
We illustrate the design goal, design considerations and design details of \textit{ARLayout} in this section.
% This part mainly discusses the design rationale of \textit{ARLayout},
% including the considerations and the details of \textit{ARLayout}.

\subsection{Design Goals}
% Based on the characteristics of ARLayout,
% combined with user’s interaction experience,

The target users of \textit{ARLayout} are the people who are not the experts in a given domain.
The ``users'' or ``target users'' in this paper refer to this group of users,
if they are not specially specified.

We summarize the following four design goals of \textit{ARLayout} according
to the requirements collected from the daily life of the target users:
% These design considerations will be refined over time under the guidance of an iterative strategy.

\begin{itemize}
\item G1: enable the target users to search/filter massive physical targets,
and then highlight the results to guide them where they are in reality.
\item G2: enable the users to re-group the physical targets according to their grouping comparison tasks.
\item G3: enable the users to re-rank/sort the physical targets according to their ordinal or quantitative comparison tasks.  %// add & cut in coding space.
\item G4: provide extra augmented information in AR environments according to their tasks.
\end{itemize}



\subsection{Design Considerations and Design Details}

We also summarize the design considerations and design details of \textit{ARLayout}
towards the design goal (G1-G4):

First, the tool should be designed to enable the target users to search/filter the massive physical targets
by one or multiple fuzzy keywords input by voice, because voice input is simple-to-use in the public's PV context,
such as a library, a bookstore, or a coffee shop.
The input method by virtual keyboard should be also provided when it is not convenient to input a voice.
The search results should be \textbf{highlighted by visual cues} to guide the users where they are in reality.
Specifically, we use flashing to highlight the search results and further
provide an AR-based \textbf{fisheye deformation design} to highlight their positions in reality.

Second, the tool should be designed to enable the users to re-group the physical targets
according to their grouping comparison tasks (\textbf{visual re-layouts}),
e.g., re-grouping them according to one/multiple of their nominal, ordinal or quantitative attributes,
because grouping can help users better compare target candidates according to the experience in our daily life.

Third, the tool should be designed to enable the users to re-rank the disordered physical targets,
or re-rank them according to one or multiple additional attributes (\textbf{visual re-layouts}).
For example, books in a library are usually sorted by classification number or index number,
which can not satisfy users' kinds of ranking needs, e.g., sorting them by the rating, price,
publisher or publish year.
Similarly, the books in a bookstore are often sorted by user groups,
more information like ratings and prices are ignored.
As a result, readers may spend a great deal of time finding an ideal book on the bookshelves.

Fourth, in people's daily life, the provided information alongside a target is usually not enough.
For example, we can see the title of a book, the name and the price of a cup of coffee.
However, the rating of a coffee, a book or a food,
and the ingredients or drinks, foods, fruits are often not provided clearly
to make comparisons.
Therefore, the tool should be designed to display extra information which is often hidden from users
or is inconvenient for them to compare.
Actually, this kind of augmented information can be provided by \textbf{juxtaposition}
or using \textbf{small multiples} to help them make a better choice.
Except for the re-layouts, the sorted information such as the price and the rating can be visualized by
\textbf{bar charts} or \textbf{line charts} in AR.

The extra textual information such as the textual ratings
or detailed textual descriptions about the current targets can be visualized
by using \textbf{word cloud} in the style of \textbf{small multiples} to show their keywords,
which are generated from online review.
Besides, it is significant that the re-layout effects
could be integrated into reality without confusion.
In addition, users may be non-experts unfamiliar with programming or AR.
The inaccurate recognition, rough interaction and tedious interface
will add their cognitive burden in a messy situation.
Therefore, \textit{ARLayout} will provide an intuitive interface
with concise interactions to avoid causing visual clutter and discomfort.


\begin{figure}[htp]
%\setlength{\abovecaptionskip}{0.05cm}
%\setlength{\belowcaptionskip}{-0.4cm}
    \centering
    \includegraphics[width=\linewidth]{images/design_overview.eps}
    \caption{
        The structural design of ARLayout,
        including simplified data processing, data transmission, and data presentation process.
    }
    \label{fig:design_overview}
\end{figure}



\subsection{Design Details: System Workflow Design}

As shown in~\autoref{fig:design_overview}, \textit{ARLayout} consists of two parts.
The first part is the mobile client, which is used to take pictures or record a real-time video and then renders objects in AR.
The second part is the server, which is employed to process almost all of the data.
The overall processing are described as follows:
The mobile client constantly takes pictures or record a real-time video of massive targets and sends them to the server.
The remote server processes those pictures or key frames, recognizing target objects in them in real-time,
and sends targets' data back to the mobile client,
which displays them in new layouts.
This workflow design has two considerations:


\textbf{Separation of heavy computing and AR presentation}:
Unlike traditional applications,
\textit{ARLayout} shifts most of the computational intensive tasks to the server.
The mobile client only needs to send the requests in multi-thread to ensure real-time target recognition.
This enables \textit{ARLayout} to handle a large amount of data
% which makes it response faster and more efficiency
without adding a heavy burden to the user’s mobile device or influencing user's interaction experience.
In the library/bookstore scenario, for example, more than \textbf{a thousand} books can be recognized in AR
with panoramic pictures.

\textbf{Separate processing of text and texture}:
The text and texture in one picture usually contains most of our desired information.
We apply different neural network to process these two kinds of data.
% Most information are contained in text or texture
% For the text and texture in the image, are combined into structured data,
% using different neural network processing.
This makes our model not only suitable for situations where information is expressed more in text,
such as library or cafe,
but also for situations where texture contains more information,
such as eyeshadows in cosmetics shops.
For more implementation details, please refer to Section~\ref{sec:implementation}.



% These three methods changes targets' attributes into position information
% to one criterion that's easily perceived by human eyes,
%  that's inefficiently perceived by human eyes to another
% which exploits the capabilities of human's visual system,
% These three methods helps
%as shown in~\autoref{fig:design_relayout}.
% e.g., re-ranking books by their prices encodes books' quantitative information to position infomation,
% enabling the user to easily compare the prices of them.


%\textbf{Searching}.
%\textit{ARLayout} allows the user to achieve
%preseted information of targets automatically in different scenarios.
%Visual feedback will be provided after each search,
%such as magnifying and highlighting the querying results.
%In this way, the targets' nominal and quantitative attributes
%are tranformed to highlight colors and scaled volumes,
%which makes the information easily perceived by the user.
%
%
%\textbf{Re-grouping}.
%\textit{ARLayout} allows the user to arbitrarily group target objects
%according to the preset dimensions in the scenario.
%Different groups are placed separately with group name shown by the side,
%e.g., book groups placed on different layers of the bookshelves, or coffee names in different areas of the menu.
%% For example, the user can choose to re-group books by their publishers,
%% authors or other attributes.
%% \textit{ARLayout} creates new bookshelves and displays different groups on different layers,
%% which encodes books' nominal information (publisher) with position.
%
%\textbf{Re-ranking}.
%\textit{ARLayout} allows the user to sort targets based on information in different dimensions.
%Sorted targets are lined up in certain order in front of the user,
%making the finding and comparing process intuitive and clear.














