\section{Evaluations}
\label{sec:evaluation}

\subsection{Case Scenarios}
We choose three typical scenarios where people often experience confusion
when finding or choosing targets in a messy situation:
(1) Find ideal books in the library or book store.
(2) Choose coffees that suits one's taste in a coffee shop
(3) Choose a suitable eyeshadow palette among many palettes in a cosmetics shop.

We discuss the usage of searching, re-grouping, and re-ranking in the three common scenarios,
as well as different dimensions concerning the three information categories mentioned in~\autoref{sec:design},
as shown in~\autoref{fig:design_compare}.
% in order to better illustrate the usage scenarios of searching, re-grouping, and re-ranking,
% we discuss how these processes embed in three common \textit{ARLayout} usage scenarios

% concerning three basic attributes in visualization: nominal, ordinal, and quantitative.

\begin{figure}[htp]
%\setlength{\abovecaptionskip}{0.05cm}
%\setlength{\belowcaptionskip}{-0.4cm}
    \centering
    \includegraphics[width=\linewidth]{images/design_compare.eps}
    \caption{
    Building re-layouts with different methods in three scenarios (library, coffee shop and eyeshadow).
    Targets in different scenarios have different kinds of attributes.
    }
    \label{fig:design_compare}
\end{figure}

In library scenario, the user can search books according to both nominal and quantitative attributes,
e.g., searching for books written by a given author, or for books at a given price;
The user can also re-group books by ordinal or quantitative attributes,
e.g., grouped by rating or by price intervals.
Books can also be re-ranked by ordinal and quantitative attributes
e.g., sorting books in ascending order based on the price.
Besides,
targets in coffee shop and eyeshadow scenarios have less information than the library scenario,
so they doesn’t involve quantitative or ordinal re-grouping.
The eyeshadows cannot be re-ranked as well due to their abstract features.






To verify the usability and effectiveness of \textit{ARLayout},
we design 5 tasks for participants to accomplish:
two in the library, two in a simulated coffee shop, and one in a simulated cosmetics shop.
The 5 tasks represent most of the possible situations happen in these 3 three places,
where people may want to choose a suitable object in a complex scene.
In accord with our 3 goals mentioned above,
the study aimed to evaluate \textit{ARLayout} regarding three aspects:
(a) whether the re-layouting effects well satisfy users' personal requirements;
(b) whether the extra information provided is useful and enough to help users;
(c) whether \textit{ARLayout} provides vivid and friendly AR interaction;


\begin{figure}[htp]
%\setlength{\abovecaptionskip}{0.05cm}
%\setlength{\belowcaptionskip}{-0.4cm}
    \centering
    \includegraphics[width=\linewidth]{images/evaluation_prestudy.eps}
    \caption{
        Pre-study result: Most of participants have experienced these scenarios in their life
        and require assistance to better obtain and organize information.
    }
    \label{fig:pre_study}
\end{figure}


\subsection{Tasks}
We use T$i$-$j$ to name the j-th task that happens in the i-th scenario.
The first two tasks are set in the library, i.e., \textbf{T1-1} and \textbf{T1-2}.
The next two tasks take place in a simulated coffee shop, i.e., \textbf{2-1} and \textbf{T2-2}.
The last task (\textbf{3-1}) take place in a simulated cosmetics shop.
% where readers usually have troublesome searching or choosing experience
% when faced with large numbers of books.
% \textbf{T1-1} is a single book searching task,
% which corresponds to readers picking a suitable book for certain purpose.
% \textbf{T1-2} is a free exploration task,
% simulating scenarios where readers walking in the library/bookstore without purpose,
% picking high-rating or interesting books to read or buy.
% The next two tasks \textbf{T2-1} and \textbf{T2-2} took place in a simulated coffee shop.
% Considering that many consumers may be unclear about different coffees' features and ingredients,
% they may have trouble selecting a suitable coffee from menu.
% \textbf{T2-1} is a comparing task, which corresponded to situations where
% consumers want to learn about and compare certain coffees.
% % , or compare several specific coffees.
% \textbf{T2-2} is a searching and choosing task that simulated situations where consumers wanted to
% choose a suitable coffee with certain criterions and personal preference.
% The last one task \textbf{T3-1} takes place in a simulated cosmetics shop, which
% simulates situations that many customers may refer to online reviews when buying cosmetics like eyeshadows,
% and need to be guided by tutorial when making up if unskilled or trying out new products.


\textbf{T1-1} required participants to find an algorithm book in front of a bookshelf.
The target book is published by Tsinghua University Press, and is suitable for code beginners.
They are asked to find it twice (without and with the help of \textit{ARLayout} respectively).
For the first time, participants need to find the book one by one, and pick up certain books
from the shelf to check the publisher and other details.
% At first, participants could use traditional method (search in the library indexing system,
% determine several alternative books, find them one by one and pick one
% with possible help of online comments and ratings).
Then they were required to use \textit{ARLayout} to pick an satisfying book.
This task simulates situations where the reader had a specific book to find,
or determine a general theme and wants to pick one that suited himself.
This task is designed to test whether \textit{ARLayout}'s
searching and re-grouping provides significant help 
in terms of time-saving and effectiveness (\textbf{G1,G2}).
% This task is designed to test whether \textit{ARLayout}'s interaction is friendly to users,
% and whether it provides significant help in terms of time-saving and effectiveness.


\textbf{T1-2} describes a situation that a casual reader walked in a library or a bookstore.
He browses bookshelves and looked for interesting books that suit his taste.
Before deciding which book to borrow or buy,
he may compare the reviewers' rating and the price of multiple selected books.
Participants are required to use \textit{ARLayout} to filter, search, and compare books during the whole process.
In this task, %is designed to be a supplement to \textbf{T1-1},
the participants are encouraged to freely generate customized layout in AR environment according to their preference.
It measures \textit{ARLayout}'s ability to fit various personal needs with customized re-group/re-rank criterions,
as well as the helpfulness of the additional information (\textbf{G2,G3,G4}).


\begin{figure}[htb]
%\setlength{\abovecaptionskip}{0.05cm}
%\setlength{\belowcaptionskip}{-0.4cm}
    \centering
    \includegraphics[width=\linewidth]{images/evaluation_result.eps}
    \caption{
        Post-study result: most of participants react positively to \textit{ARLayout}.
    }
    \label{fig:evaluation_result}
\end{figure}


\textbf{T2-1} requires participants to search for a certain kind of coffee
they may heard of before by text or speech in \textit{ARLayout}.
They cqn also browse and select other coffees that were unfamiliar to them.
They then compare and learn about those selected coffees' features and ingredients in the AR environment.
This task tests the effectiveness of searching and highlighting,
as well as whether the supplementary information is useful
for customers to learn and choose coffees (\textbf{G1,G4}).

\textbf{T2-2} requires participants to choose a coffee that suits their tastes with \textit{ARLayout}, e.g.,
a coffee with moderate milk but no sugar, or with low calories.
This task was similar with \textbf{T1-2} in the library,
testing \textit{ARLayout}'s capability of generating various customized layout 
with re-grouping and re-ranking (\textbf{G2,G3}).


\textbf{T3-1} requires participants to browse and learn about several eyeshadow's features,
as well as some tutorials with pictures in the AR environment.
They can also choose recommended collocation of different colors,
and preview the select scheme in a virtual 3-D model.
This task tests the helpfulness of searching, re-grouping, and those supplementary information to
potential buyers and makeup beginners (\textbf{G1,G2,G4}).


\subsection{User Study}

\textbf{Questionnaire.}
The study contains 29 questions which are designed by using  
five-point Likert scale form, ranging from 1 ('Not at all') to 5 ('Very much').
Besides, there are four short-answers: the first two collects participants' time spent on
finding one book with and without \textit{ARLayout} respectively in \textbf{T1-1}.
The last two record participants' most impressive task and their suggestions respectively.

\textbf{Participants.}
There are 22 participants take part in the study (11 males; age: 19 to 23)%, $\mu = 20.6$).
% The illustrated in~\autoref{fig:pre_study},
According to the pre-study,
which is illustrated in~\autoref{fig:pre_study},
many participants are not familiar with AR (Q1: $\mu = 3.27, 95\% CI = [2.80,3.75]$).
However, almost all participants encounter messy scenarios in daily lives (Q2: $\mu = 4.40, 95\% CI = [4.14,4.68]$).
They wish to refer to more information when selecting objects (Q3: $\mu = 4.59, 95\% CI = [4.35,4.83]$).
Specifically, most of them have trouble finding books
in the library where books are sorted by index number (Q4: $\mu = 4.09, 95\% CI = [3.60,4.58]$).
% with average 6.7 mins spent finding a book in the library.
% Some of them also felt confused when choosing different coffees ($\mu = 4.2$, $\sigma = 1-1$).
Some can't distinguish several coffees clearly (Q5: $\mu = 1.27, 95\% CI = [1.05,1.50]$).
Most female participants wish to have real-time virtual makeup try-on and other people's recommandations. (Q6: $\mu = 4.18, 95\% CI = [3.72,4.64]$).

\textbf{Apparatus.}
Among the five tasks, \textbf{T1-1} and \textbf{T1-2} were run in the library,
others were run in a lab with a poster (width: 1-2m; height: 2m) copied from a local coffee shop,
and five different eyeshadows. An 11-inch iPad-Pro 2020 is also provided
to participants.

% The pre-study showed most of them often encountered messy situations ($\mu = 4.3$).
% They used to wish to refer to more information when selecting objects ($\mu = 4.5$).
% Specifically, most of them had trouble finding books in the library where books are sorted by index number ($\mu = 4.0$).
% % with average 6.7 mins spent finding a book in the library.
% % Some of them also felt confused when choosing different coffees ($\mu = 4.2$, $\sigma = 1-1$).
% Some of them couldn't distinguish several coffees clearly ($\mu = 2.47$).
% % or buying eyeshadows ($\mu = 4.4$, $\sigma = 1.0$).
% Among participants, all 11 female ones have been used eyeshadows, while all male ones
% never tried eyeshadows. However, 6 male participants had experienced confusion
% when buying cosmetics for family or friends.
% These 17 participants felt it difficult to choose a suitable eyeshadow($\mu = 4.4$)


\textbf{Procedures.}
Participants are required to first fill in the pre-study questions.
Before entering each scene (before \textbf{T1-1}, \textbf{T2-2} and \textbf{T3-1}), we spend 3-5 mins introducing
different functions and re-layout effects, and show the operation with an actual example.
Participants are then given tasks to complete.
Finally, participants fill in the post-study questions.


\subsection{User Study Results}
The results are shown in~\autoref{fig:evaluation_result}.
We analyse questionnaires from five aspects: usability, expressiveness, effectiveness,
involvement and other suggestions.

\textbf{Usability.} As we designed \textit{ARLayout} to help users reduce complexity,
% itself should not have a steep learning curve in the first place.
its usage should be simple enough in the first place.
According to our study,
most participants give high scores to the overall interaction (Q1: $\mu = 4.31, 95\% CI = [4.03,4.61]$).
In particular, in Q2 ($\mu = 4.50, 95\% CI = [4.17,4.83]$), participants feel UI operation easy to understand,
in Q3 ($\mu = 4.36, 95\% CI = [4.10,4.63]$), voice input is recognized as helpful and efficient,
and in Q4 ($\mu = 4.36, 95\% CI = [4.00,4.73]$), fisheye deformation and result highlighting
make the AR interactions clear and intuitive.

Since \textit{ARLayout} has different functions and interactions in three scenarios, it receives high praise
in library/bookstore tasks (Q5: $\mu = 4.50, 95\% CI = [4.29,4.71]$) and coffee shop tasks (Q6: $\mu = 4.54, 95\% CI = [4.34,4.75]$),
while receives lower scores in the eyeshadow task (Q7: $\mu = 3.58, 95\% CI = [3.07,4.11]$).
When asked, participants say its' better to have eyeshadow video tutorials rather than just pictures.

\textbf{Expressiveness.} \textit{ARLayout} provides augmented information for messy objects, so it's necessary
that extra information and AR effects is expressive instead of adding to user's cognitive burden.
According to Q8 ($\mu = 4.68, 95\% CI = [4.49,4.88]$) and Q9 ($\mu = 4.40, 95\% CI = [4.14,4.68]$),
bar charts and word clouds concisely help participants obtain a general grasp of objects.
Participants also rate that coffee component graphs quickly give them overall impressions about
certain coffees in Q10 ($\mu = 4.41, 95\% CI = [4.04,4.78]$).

\textbf{Effectiveness.} As for the most primary functions of \textit{ARLayout},
participants respond positively and confirm the effectiveness of searching and filtering (Q11: $\mu = 4.73, 95\% CI = [,]$),
re-grouping (Q12: $\mu = 4.73, 95\% CI = [4.54,4.91]$) and re-ranking (Q13: $\mu = 4.55, 95\% CI = [4.30,4.79]$).

More specifically, compared with finding books traditionally,
time cost reduced from average 1.35 minutes to less than 20 seconds after using \textit{ARLayout}.
P2 who used to be a temporal librarian, spends 5 seconds traditionally (the fastest),
and 12 seconds with \textit{ARLayout}.
% The fastest (P2 who used to be a temporal librarian) spent 30 seconds traditionally,
We revisit him and he say ``\textit{ARLayout} is useful for the public,
but there's room for improvement with UI tips''.

According to Q15 ($\mu = 4.68, 95\% CI = [4.49,4.88]$),
most participants find the browsing and choosing coffee process
helpful to them as they don't know much about coffee.
P7 says ``It helps me especially when I pay attention to fat intake''.

In terms of eyeshadows, most participants find re-grouping by rating is effective,
and were willing to buy or try the high-rating eyeshadows(Q17: $\mu = 4.41, 95\% CI = [4.13,4.70]$).


\textbf{Involvement.} As indicated by Q18 ($\mu = 4.65, 95\% CI = [4.66,4.98]$), almost all participants
feel concentrated when doing tasks, and consider the process quite smooth and interesting.


\textbf{Other Suggestions.} In addition, constructive suggestions and other responses are collected in the study,
which are mentioned below:

P6 notes: ``The fish-eye effects of books make them overlapped and cluttered''.
Considering the density of books on the shelves in the library/bookstore,
replacing the current effects with
magnifying as well as pushing away nearby books may be a future improvement.

Besides, in the eyeshadow case, we find female eyeshadow users give lower scores in Q16($\mu = 3.45$)
compared to male buyers ($\mu = 4.33$). This is probably because
\textit{ARLayout} recognizes eyeshadows merely by color, ignoring detailed texture and eyeshadows' brand which
may be considered to be fair important factors. As male buyers just care about
eyeshadow ratings, female users like P21 find it ``not so useful as detailed texture effects
can't be shown correctly on the preview 3-D model''. We may consider advancing the recognizing
algorithm in the future to distinguish among shimmery ones, matte ones and other kinds of eyeshadows.

\begin{figure}[htp]
%\setlength{\abovecaptionskip}{0.05cm}
%\setlength{\belowcaptionskip}{-0.4cm}
    \centering
    \includegraphics[width=\linewidth]{images/implementation_ocr.eps}
    \caption{
        % These images show the difficulties of traditional OCR and the corresponding improvements.
        The books in (a) is slanted,
        so the OCR module is checked for direction and corrected before it is actually called.
        The final result is figure (b).
        (c) and (d) show the difference in the arrangement of text between Chinese and English books.
        Artistic fonts exist in (e).
        % The non-uniform width and the existence of special deformation of the font,
        %sometimes affect the accuracy of recognition. Therefore, when recognition is difficult,
        %we try to infer the content of the book from the rest of the text in the spine.
        (f) and (g) define the reference regions required for a custom template.
        Reference Fields are identified in (f) and Identification Areas are identified in (g).
    }
    \label{fig:implementation_ocr}
\end{figure}

\begin{figure*}[htp]
    % \setlength{\abovecaptionskip}{0.05cm}
    % \setlength{\belowcaptionskip}{-0.4cm}
        \centering
        \includegraphics[width=\linewidth]{images/implementation_network_deeplab.eps}
        \caption{
            The network illustration about how PaddleSeg is integrated into \textit{ARLayout}.
            We take DeepLab as an example, one of the key modules of PaddleSeg.
            The encoder module encodes multi-scale contextual information by
            applying atrous convolution at multiple scales, while the simple yet effective decoder
            module refines the segmentation results along object boundaries.
        }
        \label{fig:implementation_network_deeplab}
    \end{figure*}