\section{Conclusion}

%In this work, we present \textit{ARLayout},
%an authoring tool designed for non-experts
%to build AR-based visual re-layouts over massive physical targets.
%The design and implementation of \textit{ARLayout} is based on the C/S architecture.
%OCR and image segmentation are used to process the text and image data in the scenario,
%and AR + VR (mainly AR) are used to display the data.
%We demonstrate the expressiveness, usability and usefulness of \textit{ARLayout}
%through three usage scenario examples and a user study.


In this paper, we present \textit{ARLayout}, a personal visualization tool designed
for non-experts to build AR-based visual re-layouts towards massive physical targets,
such as books in a library/bookstore, coffees in a coffee shop, eyeshadows in a shop, etc.
The real-time video are captured from the reality world by the camera of
mobile devices.
All the candidate targets can be segmented and labeled by
a convolutional neural network PaddleSeg~\cite{Liu2019,Liu2021a}.
The network just requires a small scale of labeled training
samples, i.e., less than 100 for each case in our experiments,
while the textual information is recognized by an OCR algorithm.

The visual re-layouts of the physical targets include (1) highlighting
the search results by AR-based fisheye deformation,
(2) re-grouping them according to their additional nominal,
ordinal or quantitative attributes,
and (3) re-ranking them according to their additional ordinal or quantitative attributes.
In the searching scenario, the fuzzy keywords are input by voice to narrow down the number of candidate targets,
and then the search results will be highlighted by flashing, transparency or AR-based fisheye deformation
in the AR environment to guide them where to find the targets in the reality world.
In the re-grouping task, candidate targets can be re-grouped in the AR environment
according to one or multiple of their nominal, ordinal or quantitative attributes.
In the re-ranking task, candidate targets can be sorted in the AR environment according to one of
their ordinal or quantitative attributes.
In the experiments, we demonstrate the usability, expressiveness and effectiveness
of \textit{ARLayout} by a user study and three case studies.
